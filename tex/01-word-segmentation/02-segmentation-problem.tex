\item \points{1b}
Implement an algorithm that, unlike the greedy algorithm, finds the optimal word
segmentation of an input character sequence. Your algorithm will consider costs
based simply on a unigram cost function. {\tt UniformCostSearch} (UCS) is
implemented for you in {\tt util.py}, and you should make use of it here.
\footnote{Solutions that use UCS ought to exhibit fairly fast execution time for
this problem, so using A* here is unnecessary.}

Before jumping into code, you should think about how to frame this problem as a
state-space {\bf search problem}.  How would you represent a state?  What are
the successors of a state?  What are the state transition costs?  (You don't
need to answer these questions in your writeup.)

Fill in the member functions of the {\tt SegmentationProblem} class and the
{\tt segmentWords} function.

The argument {\tt unigramCost} is a function that takes in a single string
representing a word and outputs its unigram cost. You can assume that all of the
inputs would be in lower case.

The function {\tt segmentWords} should return the segmented sentence with spaces
as delimiters, i.e. {\tt ' '.join(words)}.

For convenience, you can actually run {\tt python submission.py} to enter a
console in which you can type character sequences that will be segmented by your
implementation of {\tt segmentWords}.  To request a segmentation, type
{\tt seg mystring} into the prompt.  For example:

\begin{lstlisting}
>> seg thisisnotmybeautifulhouse

Query (seg): thisisnotmybeautifulhouse

this is not my beautiful house
\end{lstlisting}

Console commands other than |seg| --- namely |ins| and |both| --- will be used
in the upcoming parts of the assignment.  Other commands that might help with
debugging can be found by typing |help| at the prompt.

{\em Hint: You are encouraged to refer to |NumberLineSearchProblem| and
|GridSearchProblem| implemented in |util.py| for reference. They don't
contribute to testing your submitted code but only serve as a guideline for what
your code should look like.}

{\em Hint: The actions that are valid for the |ucs| object can be
accessed through |ucs.actions|.}
